\documentclass[a4paper,footrule,landscape]{foils}

\usepackage{jmbr-traspas}
\usepackage[latin1]{inputenc}
\usepackage[spanish]{babel}
\usepackage{amsmath,hhline,fancybox,pifont}
\usepackage[noend]{algoritmico}
\usepackage{epsfig,epic,ecltree,alltt}
%%%%%%%%%%%%%%%%%%%%%%%%%%%%%%%%%%%%%%%%%%%%%%%%%%%%%%%%%%%%%%%%%%%%%%%%
%%%% DOCUMENT0
%%%%%%%%%%%%%%%%%%%%%%%%%%%%%%%%%%%%%%%%%%%%%%%%%%%%%%%%%%%%%%%%%%%%%%%%
\begin{document}
%%%%%%%%%%%%%%%%%%%%%%%%%%%%%%%%%%%%%%%%%%%%%%%%%%%%%%%%%%%%%%%%%%%%%%%%
%%%%%%%%%%%%%%%%%%%%%%%%%%%%%%%%%%%%%%%%%%%%%%%%%%%%%%%%%%%%%%%%%%%%%%%%
%%%% Definici�n inicial de los bordes 
%%%%%%%%%%%%%%%%%%%%%%%%%%%%%%%%%%%%%%%%%%%%%%%%%%%%%%%%%%%%%%%%%%%%%%%%
\rightheader{} \leftheader{} \rightfooter{} \MyLogo{}
\Restriction{}
%%%%%%%%%%%%%%%%%%%%%%%%%%%%%%%%%%%%%%%%%%%%%%%%%%%%%%%%%%%%%%%%%%%%%%%%
%% Cabecera
%%%%%%%%%%%%%%%%%%%%%%%%%%%%%%%%%%%%%%%%%%%%%%%%%%%%%%%%%%%%%%%%%%%%%%%%
\title{\vspace{10mm}\textbf{\textcolor{blue}
   {LENGUAJES DE PROGRAMACI�N \\
    Y\\
   PROCESADORES DE LENGUAJES}}}
 
\author{\vspace{5mm} \mbox{}\\ 
  {\Large Construcci�n de un compilador}}

\date{\vspace{10mm}\mbox{}\\
\textcolor{blue}{\LARGE \texttt{MenosC}}
\vspace{25mm}\mbox{}\\
{\Large Parte-I: An�lisis l�xico sint�ctico}}
%%%%%%%%%%%%%%%%%%%%%%%%%%%%%%%%%%%%%%%%%%%%%%%%%%%%%%%%%%%%%%%%%%%%%%%%
\maketitle
%%%%%%%%%%%%%%%%%%%%%%%%%%%%%%%%%%%%%%%%%%%%%%%%%%%%%%%%%%%%%%%%%%%%%%%%
\setcounter{page}{0} 
\rightfooter{Lenguaje de Programaci�n y Procesadores de
  Lenguajes~~\textsf{\thepage}}
%%%%%%%%%%%%%%%%%%%%%%%%%%%%%%%%%%%%%%%%%%%%%%%%%%%%%%%%%%%%%%%%%%%%%%%%
\begin{traspa}{270}{construcci�n de un compilador}

\vspace{2mm}\noindent%
El objetivo principal es la \textcolor{blue}{construcci�n de un
  compilador} completo, \\[4mm]
para un Lenguaje de Programaci�n de alto nivel, sencillo pero no
trivial

\vspace{14mm}\hspace{9cm}
\texttt{\textcolor{blue}{\Large MenosC}}

\vspace{15mm}
\begin{tabular}{lp{22cm}}
\texttt{Parte~~~I}& Construcci�n del analizador 
  l�xico-sint�ctico\\[4mm]
&\multicolumn{1}{r}{$\Rightarrow$ ~~l�mite de entrega 
  \textbf{~~~8 de noviembre de 2020}}\\[7mm]
\texttt{Parte~~II}& Construcci�n del analizador 
  sem�ntico\\[4mm]
&\multicolumn{1}{r}{$\Rightarrow$ ~l�mite de entrega 
  \textbf{~~~6 de diciembre de 2020}}\\[7mm]
\texttt{Parte~III}& Construcci�n del generador 
  de c�digo intermedio\\[4mm]
&\multicolumn{1}{r}{$\Rightarrow$ ~l�mite de entrega 
  \textbf{~15 de enero de 2021}}\\[4mm]
&\multicolumn{1}{r}{[~para la recuperaci�n:
  \textbf{~~27 de enero de 2021}~]}
\end{tabular}

\end{traspa}
%%%%%%%%%%%%%%%%%%%%%%%%%%%%%%%%%%%%%%%%%%%%%%%%%%%%%%%%%%%%%%%%%%%%%%%%
\begin{traspa}{270}{Especificaci�n L�xica}

\vspace{0mm}
\begin{blistas}{0}{2}
\item Los identificadores son cadenas de letras (incluyendo
  ``\texttt{\_}'') y d�gitos, \\[2mm]
  que comienzan siempre por una letra.  Debe distinguirse
  entre\\[2mm] may�sculas y min�sculas.
\item Las palabras reservadas se deben escribir en min�scula.
\item En un programa fuente puedan aparecer constantes enteras y
  reales; \\[2mm]
  por ejemplo: \texttt{~~28 ~~28. ~~.55 ~~28.55}
\item El signo $+$ (� $-$) de las constantes num�ricas se tratar� 
  como\\[2mm]
  un s�mbolo l�xico independiente.
\item Los espacios en blanco, retornos de linea y tabuladores deben
  ignorarse.
\item Los comentarios deben ir precedidos por la doble barra
  (\texttt{//}) y terminar \\[2mm]
  con el fin de la linea.  Los comentarios no se pueden anidar.
\end{blistas}

\end{traspa}
%%%%%%%%%%%%%%%%%%%%%%%%%%%%%%%%%%%%%%%%%%%%%%%%%%%%%%%%%%%%%%%%%%%%%%%%
\begin{traspa}{270}{Especificaci�n Sint�ctica}

\vspace{-3mm}
\begin{center}
\begin{tabular}{|l@{ }c@{ }l|}\hline
&&\\[-4mm]
programa                      &$\rightarrow$&  
  listaDeclaraciones
\\[3mm]
listaDeclaraciones            &$\rightarrow$& 
  declaracion $\;|\;$ listaDeclaraciones  ~declaracion
\\[3mm]
declaracion                   &$\rightarrow$&  
  declaracionVariable $\;|\;$ declaracionFuncion
\\[3mm]
declaracionVariable           &$\rightarrow$&   
  tipoSimple {\bf id} ; $\;|\;$ tipoSimple {\bf id} [ {\bf cte} ] ;
\\[3mm]
tipoSimple                    &$\rightarrow$&  
  {\bf int} $\;|\;$ {\bf bool}
\\[3mm]
declaracionFuncion            &$\rightarrow$&
   cabeceraFuncion ~bloque
\\[3mm]
cabeceraFuncion               &$\rightarrow$&
  tipoSimple {\bf id} ( parametrosFormales )
\\[3mm]
parametrosFormales            &$\rightarrow$&
  $\epsilon$ $\;|\;$ listaParametrosFormales
\\[3mm]
listaParametrosFormales       &$\rightarrow$&
  tipoSimple  {\bf id} $\;|\;$ 
  tipoSimple  {\bf id} , listaParametrosFormales
\\[3mm]  
bloque                        &$\rightarrow$&
  \{ declaracionVariableLocal ~listaInstrucciones 
\\[3mm]  
&& {\bf return} ~expresion ; \}
\\[3mm]       
declaracionVariableLocal      &$\rightarrow$&
  $\epsilon$ $\;|\;$ declaracionVariableLocal ~declaracionVariable
\\[2mm]\hline       
\end{tabular}
\end{center}

\end{traspa}
%%%%%%%%%%%%%%%%%%%%%%%%%%%%%%%%%%%%%%%%%%%%%%%%%%%%%%%%%%%%%%%%%%%%%%%%
\begin{traspa}{270}{Especificaci�n Sint�ctica}

\vspace{-3mm}
\hspace{-5mm}\begin{tabular}{|l@{ }c@{ }l|}\hline
&&\\[-4mm]
listaInstrucciones            &$\rightarrow$& 
  $\epsilon$ $\;|\;$ listaInstrucciones ~instruccion
\\[3mm]
instruccion                   &$\rightarrow$& 
  \{ listaInstrucciones \} 
  $\;|\; $ instruccionAsignacion
\\[3mm]
&&$\;|\;$ instruccionSeleccion $\;|\; $ instruccionEntradaSalida 
\\[3mm]
&&$\;|\;$  instruccionIteracion
\\[3mm]
instruccionAsignacion         &$\rightarrow$&  
  {\bf id} = expresion ;
  $\;|\;$ {\bf id} ~[ ~expresion ~] = expresion ;
\\[3mm]
instruccionEntradaSalida      &$\rightarrow$& 
  \textbf{read} ( \textbf{id} ) ; 
  $\;|\; $ \textbf{print} ( expresion ) ;
\\[3mm]
instruccionSeleccion          &$\rightarrow$& 
  \textbf{if} ( expresion ) instruccion {\bf else} ~instruccion
\\[3mm]
instruccionIteracion         &$\rightarrow$& 
  {\bf for} ( expresionOpcional ; expresion ; 
  expresionOpcional )
\\[3mm]
&& instruccion 
\\[3mm]
expresionOpcional            &$\rightarrow$& 
  $\epsilon$ $\;|\;$ expresion $\;|\;$ {\bf id} = expresion
\\[3mm]
expresion                    &$\rightarrow$& 
  expresionIgualdad                     \\[1mm]
 && $\; |\;$ expresion ~operadorLogico ~~expresionIgualdad     
\\[2mm]\hline
\end{tabular}

\end{traspa}
%%%%%%%%%%%%%%%%%%%%%%%%%%%%%%%%%%%%%%%%%%%%%%%%%%%%%%%%%%%%%%%%%%%%%%%%
\begin{traspa}{270}{Especificaci�n Sint�ctica}

\vspace{-3mm}
\begin{center}
\begin{tabular}{|l@{ }c@{ }l|}\hline
&&\\[-4mm]
expresionIgualdad     &$\rightarrow$& expresionRelacional            
\\[3mm]
 && $\;|\;$ expresionIgualdad ~operadorIgualdad ~expresionRelacional 
\\[3mm]
expresionRelacional     &$\rightarrow$& expresionAditiva             
\\[3mm]
 && $\;|\;$ expresionRelacional ~operadorRelacional ~expresionAditiva
\\[3mm]
expresionAditiva        &$\rightarrow$& expresionMultiplicativa      
\\[3mm]
 && $\;|\;$ 
    expresionAditiva ~operadorAditivo ~expresionMultiplicativa       
\\[3mm]
expresionMultiplicativa &$\rightarrow$& expresionUnaria              
\\[3mm]
 && $\;|\;$ 
    expresionMultiplicativa operadorMultiplicatico 
\\[3mm]
&& $\quad$ expresionUnaria 
\\[3mm]
expresionUnaria         &$\rightarrow$&  expresionSufija 
 $\;|\;$ operadorUnario ~expresionUnaria                             
\\[3mm]
 && $\;|\;$ operadorIncremento {\bf id}                              
\\[3mm]
expresionSufija         &$\rightarrow$&
  ( expresion ) $\;|\;$ {\bf id} operadorIncremento  
  $\;|\;$ {\bf id} [ expresion ]                                     
\\[3mm]
 && $\;|\;$ {\bf id} ( parametrosActuales ) $\;|\;$ 
    {\bf id} $\;|\;$ constante
\\[2mm]\hline
\end{tabular}
\end{center}

\end{traspa}
%%%%%%%%%%%%%%%%%%%%%%%%%%%%%%%%%%%%%%%%%%%%%%%%%%%%%%%%%%%%%%%%%%%%%%%%
\begin{traspa}{270}{Especificaci�n Sint�ctica}

\vspace{-3mm}
\begin{center}
\begin{tabular}{|l@{ }c@{ }l|}\hline
&&\\[-4mm]
parametrosActuales            &$\rightarrow$& 
  $\epsilon$ $\;|\;$ listaParametrosActuales
\\[3mm] 
listaParametrosActuales        &$\rightarrow$& 
  expresion $\;|\;$ expresion , listaParametrosActuales
\\[3mm] 
constante  &$\rightarrow$& 
  {\bf cte} $\; |\; $ {\bf true} $\; |\; $ {\bf false}               
\\[3mm]
operadorLogico          &$\rightarrow$&       
  \texttt{~\&\&} $\;\;|\;\;$ \texttt{||}                         
\\[3mm]
operadorIgualdad      &$\rightarrow$& 
  \texttt{~==} $\;\;|\;\;$ \texttt{!=}                           
\\[3mm]
operadorRelacional      &$\rightarrow$& 
  \texttt{~>}  $\;\;|\;\;$ \texttt{<}  $\;\;|\;\;$ 
  \texttt{>=} $\;\;|\;\;$ \texttt{<=}                            
\\[3mm]
operadorAditivo         &$\rightarrow$& 
  \texttt{~+} $\;\;|\;\;$ \texttt{-}                             
\\[3mm]
operadorMultiplicativo  &$\rightarrow$& 
  \texttt{~*} $\;\;|\;\;$ \texttt{/}                             
\\[3mm]
operadorUnario           &$\rightarrow$& 
  \texttt{~+} $\;\;|\;\;$ \texttt{-} $\;\;|\;\;$ \texttt{!}      
\\[3mm]
operadorIncremento           &$\rightarrow$& 
  \texttt{~++} $\;\;|\;\;$ \texttt{--}                           
\\[2mm]\hline
\end{tabular}
\end{center}

\end{traspa}
%%%%%%%%%%%%%%%%%%%%%%%%%%%%%%%%%%%%%%%%%%%%%%%%%%%%%%%%%%%%%%%%%%%%%%%%
%%%%%%%%%%%%%%%%%%%%%%%%%%%%%%%%%%%%%%%%%%%%%%%%%%%%%%%%%%%%%%%%%%%%%%%%
\end{document}
%%%%%%%%%%%%%%%%%%%%%%%%%%%%%%%%%%%%%%%%%%%%%%%%%%%%%%%%%%%%%%%%%%%%%%%%
