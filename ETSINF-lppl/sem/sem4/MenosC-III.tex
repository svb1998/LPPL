\documentclass[a4paper,footrule,landscape]{foils}

\usepackage{jmbr-traspas}
\usepackage[latin1]{inputenc}
\usepackage[spanish]{babel}
\usepackage{amsmath,hhline,fancybox,pifont}
\usepackage[noend]{algoritmico}
\usepackage{epsfig,epic,ecltree,alltt}
%%%%%%%%%%%%%%%%%%%%%%%%%%%%%%%%%%%%%%%%%%%%%%%%%%%%%%%%%%%%%%%%%%%%%%%%
%%%% DOCUMENT0
%%%%%%%%%%%%%%%%%%%%%%%%%%%%%%%%%%%%%%%%%%%%%%%%%%%%%%%%%%%%%%%%%%%%%%%%
\begin{document}
%%%%%%%%%%%%%%%%%%%%%%%%%%%%%%%%%%%%%%%%%%%%%%%%%%%%%%%%%%%%%%%%%%%%%%%%
%%%%%%%%%%%%%%%%%%%%%%%%%%%%%%%%%%%%%%%%%%%%%%%%%%%%%%%%%%%%%%%%%%%%%%%%
%%%% Definici?n inicial de los bordes 
%%%%%%%%%%%%%%%%%%%%%%%%%%%%%%%%%%%%%%%%%%%%%%%%%%%%%%%%%%%%%%%%%%%%%%%%
\rightheader{} \leftheader{} \rightfooter{} \MyLogo{}
\Restriction{}
%%%%%%%%%%%%%%%%%%%%%%%%%%%%%%%%%%%%%%%%%%%%%%%%%%%%%%%%%%%%%%%%%%%%%%%%
%% Cabecera
%%%%%%%%%%%%%%%%%%%%%%%%%%%%%%%%%%%%%%%%%%%%%%%%%%%%%%%%%%%%%%%%%%%%%%%%
\title{\vspace{10mm}\textbf{\textcolor{blue}
   {LENGUAJES DE PROGRAMACI�N \\
    Y\\
   PROCESADORES DE LENGUAJES}}}
 
\author{\vspace{5mm} \mbox{}\\ 
  {\Large Construcci�n de un compilador}}

\date{\vspace{10mm}\mbox{}\\
\textcolor{blue}{\LARGE \texttt{MenosC}}
\vspace{25mm}\mbox{}\\
{\Large Parte-III: Generaci�n de C�digo Intermedio}}
%%%%%%%%%%%%%%%%%%%%%%%%%%%%%%%%%%%%%%%%%%%%%%%%%%%%%%%%%%%%%%%%%%%%%%%%
\maketitle
%%%%%%%%%%%%%%%%%%%%%%%%%%%%%%%%%%%%%%%%%%%%%%%%%%%%%%%%%%%%%%%%%%%%%%%%
\setcounter{page}{0} 
\rightfooter{Lenguaje de Programaci�n y Procesadores de
  Lenguajes~~\textsf{\thepage}}
%%%%%%%%%%%%%%%%%%%%%%%%%%%%%%%%%%%%%%%%%%%%%%%%%%%%%%%%%%%%%%%%%%%%%%%%
\begin{traspa}{270}{construcci�n de un compilador}

\vspace{0mm}
\noindent\textcolor{blue}{\Large Material de pr�cticas} 

\begin{blistas}{7}{3}
\item \textcolor{blue}{\texttt{\large Makefile}}. ~Una nueva versi�n.
\item \textcolor{blue}{\texttt{\large principal.c}}. ~Una nueva
  versi�n en el directorio \texttt{\bf ~src}.

\item \textcolor{blue}{\large \texttt{libgci}} ~Librer�a para
  facilitar la tarea de generaci�n de c�digo intermedio.

  \vspace{5mm}\hspace{10mm}%
  \textcolor{blue}{\texttt{libgci.h}}, ~el fichero de cabecera, en el
  directorio \texttt{\bf ~include};

  \vspace{5mm}\hspace{10mm}%
  \textcolor{blue}{\texttt{libgci.a}}, ~la librer�a, en el directorio
  \texttt{\bf ~lib}.

\item \textcolor{blue}{\large \texttt{mvm}}.- M�quina virtual
  que permite ejecutar el c�digo intermedio \texttt{~Malpas}

  Est� disponible en el directorio \textbf{bin}.

\item \textcolor{blue}{\large {\em Programas de prueba}} ~En el
  directorio \textbf{tmp}
\end{blistas}

\end{traspa}
%%%%%%%%%%%%%%%%%%%%%%%%%%%%%%%%%%%%%%%%%%%%%%%%%%%%%%%%%%%%%%%%%%%%%%%%
\begin{traspa}{270}{{\sc Malpas}: inventario de instrucciones}

\vspace{-4mm}
\begin{tabular}{@{}c@{}c}
\begin{minipage}[t]{.5\textwidth}
\centerline{\textcolor{blue}{Operaciones aritm�ticas}}

\vspace{4mm}
\begin{footnotesize}
\begin{tabular}{|c|c|c|c|l|}\hline 
OP   &arg1&arg2&res&Significado\\\hline\hline 
ESUM &I/P &I/P &P  &Suma                     \\\hline 
EDIF &I/P &I/P &P  &Resta                    \\\hline 
EMULT&I/P &I/P &P  &Multiplicaci�n           \\\hline 
EDIVI&I/P &I/P &P  &Divisi�n entera          \\\hline 
RESTO&I/P &I/P &P  &Resto divisi�n entera    \\\hline 
ESIG &I/P &    &P  &Cambio de signo          \\\hline 
EASIG&I/P &    &P  &Asignaci�n               \\\hline
\end{tabular}
\end{footnotesize}

\vspace{10mm}
\centerline{\textcolor{blue}{Operaciones de entrada/salida}}

\vspace{4mm}
\begin{footnotesize}
\begin{tabular}{|c|c|c|c|p{4cm}|}\hline 
OP    &arg1&arg2&arg3&Significado\\\hline\hline 
EREAD &    &    &P   &Lectura  \\\hline 
EWRITE&    &    &I/P &Escritura\\\hline
\end{tabular}
\end{footnotesize}
\end{minipage}
&
\begin{minipage}[t]{.5\textwidth}
\centerline{\textcolor{blue}{Operaciones de salto}}

\vspace{4mm}
\begin{footnotesize}
\begin{tabular}{|c|c|c|c|l|}\hline 
OP    &arg1&arg2&res&Significado\\\hline\hline 
GOTOS &    &    &E  &Salto incondicional a E  \\\hline 
EIGUAL&I/P &I/P &E  &si arg1$=$arg2 salto a E \\\hline 
EDIST &I/P &I/P &E  &si arg1$<>$arg2 salto a E\\\hline 
EMEN  &I/P &I/P &E  &si arg1$<$arg2 salto a E \\\hline 
EMAY  &I/P &I/P &E  &si arg1$>$arg2 salto a E \\\hline 
EMENEQ&I/P &I/P &E  &si arg1$<=$arg2 salto a E\\\hline 
EMAYEQ&I/P &I/P &E  &si arg1$>=$arg2 salto a E\\\hline
FIN   &    &    &   &Fin del programa\\\hline
\end{tabular}
\end{footnotesize}
\end{minipage}
\end{tabular}

\vspace{10mm} 
\centerline{\textcolor{blue}{Operaciones con direccionamiento relativo
    (vectores)}}

\vspace{4mm}
\begin{footnotesize}
\begin{tabular}{|c|c|c|c|p{19cm}|}\hline 
OP &arg1&arg2&res&Significado\\\hline\hline 
EAV&P   &I/P &P  &Asigna un elemento de un vector a una variable: 
   $\; res:=arg1[arg2]$\\\hline 
EVA&P   &I/P &P  &Asigna una variable a un elemento de un vector: 
   $\; arg1[arg2]:=res$\\\hline
\end{tabular}
\end{footnotesize}

\end{traspa}

%%%%%%%%%%%%%%%%%%%%%%%%%%%%%%%%%%%%%%%%%%%%%%%%%%%%%%%%%%%%%%%%%%%%%%%%
\begin{traspa}{270}{{\sc Malpas}: inventario de instrucciones}

\vspace{-5mm}
\centerline{\textcolor{blue}{Operaciones de llamada}}
%
\vspace{4mm}
\begin{center}
\begin{footnotesize}
\begin{tabular}{|c|c|c|c|p{8.5cm}|}\hline 
OP  &arg1&arg2&res&Significado\\\hline\hline 
FIN &    &    &   &Fin del programa\\\hline 
RET &    &    &   &Desapila la direcci�n de retorno y transfiere 
                   el control a dicha direcci�n\\\hline 
CALL&    &    &E  &Apila la direcci�n de retorno y transfiere el\\
    &    &    &   &      control a $\,res$\\\hline
\end{tabular}
\end{footnotesize}
\end{center}

\vspace{10mm}
\centerline{\textcolor{blue}{Operaciones de manejo de pila de RA}}
%
\vspace{4mm}
\begin{center}
\begin{footnotesize}
\begin{tabular}{|c|c|c|c|p{9.9cm}|}\hline 
OP    &arg1&arg2&res&Significado\\\hline\hline 
EPUSH &    &    &I/P&Apila $\,res\,$ en la cima de la pila\\\hline 
EPOP  &    &    &P  &Desapila la cima de la pila y la deposita 
                     en $\,res$\\\hline 
PUSHFP&    &    &   &Apila el {\sc fp} en la cima de la 
                     pila \\\hline 
FPPOP &    &    &   &Desapila la cima y la deposita en el {\sc fp}
\\\hline
FPTOP &    &    &   &El {\sc fp} apunta a la misma posici�n que la cima 
                     de la pila\\\hline 
TOPFP &    &    &   &La cima de la pila apunta a la misma posici�n que
                     el {\sc fp}\\\hline 
INCTOP&    &    &I  &Incrementa la cima de la pila en $\,res$ 
                     posiciones\\\hline 
DECTOP&    &    &I  &Decrementa la cima de la pila en $\,res$ 
                     posiciones\\\hline
\end{tabular}
\end{footnotesize}
\end{center}

\end{traspa}
%%%%%%%%%%%%%%%%%%%%%%%%%%%%%%%%%%%%%%%%%%%%%%%%%%%%%%%%%%%%%%%%%%%%%%%%
\begin{traspa}{270}{{\sc Malpas}: arquitectura}

\vspace{0mm}
\footnotesize
\begin{center}
\setlength{\unitlength}{5mm}
\begin{picture}(35,30)
  \put(13,4){\framebox(11,3){\small Seg.Var.Globales}}
  \put(13,7){\makebox(11,3){$\ldots$}}
  \put(13,10){\framebox(11,3){\small Valor de retorno}}
  \put(13,13){\framebox(11,3){\small Seg. Par�metros}}
  \put(13,16){\framebox(11,3){\textcolor{blue}{\small Dir. Retorno}}}
  \put(13,19){\framebox(11,3){\textcolor{blue}{\small Enlaces Control}}}
  \put(13,22){\framebox(11,3){\small Seg. Var. locales}}
  \put(13,25){\framebox(11,3){\small Seg. Var. temporales}}
    \put(8.25,27.8){\makebox{\small \textbf{sp}}} 
    \put(11,28){\vector(1,0){2}}
    \put(9.25,21.8) {\makebox{\textbf{fp}}} 
    \put(11,4){\vector(1,0){2}}
    \put(4.5,4) {\makebox{\texttt{OrSegGlobal}}} 

    \put(11,22){\vector(1,0){2}}
    \put(24.5,14){\makebox(10,10){$
    \left\}\rule[-4mm]{0mm}{50mm}\right.$ {\small RA funci�n actual}}} 
\end{picture}
\end{center}
\normalsize

\end{traspa}
%%%%%%%%%%%%%%%%%%%%%%%%%%%%%%%%%%%%%%%%%%%%%%%%%%%%%%%%%%%%%%%%%%%%%%%%
\begin{traspa}{270}{{\sc Generaci�n de C�digo Intermedio}}

\vspace{-10mm}
\begin{blistas}{0}{0}
\item \textcolor{blue}{\large Estructura de la librer�a \texttt{libgci}}

  Constantes, variables globales y estructuras
  b�sicas\hfill (ver Secci�n \texttt{10.1} del Enunciado)

\item \textcolor{blue}{\large Funciones de para la GCI}

\begin{small}
\begin{alltt}
\textcolor{red}{Funciones generales}
void \textcolor{blue}{emite} (int {\bf cop}, TIPO_ARG {\bf arg1}, TIPO_ARG {\bf arg2}, TIPO_ARG {\bf res});
int \textcolor{blue}{creaVarTemp} ();
void \textcolor{blue}{vuelcaCodigo} (char {\bf *nom});

\textcolor{red}{Funciones para crear los argumentos de las instrucciones}
TIPO_ARG \textcolor{blue}{crArgNul} ();
TIPO_ARG \textcolor{blue}{crArgEnt} (int {\bf valor});
TIPO_ARG \textcolor{blue}{crArgEtq} (int {\bf valor});
TIPO_ARG \textcolor{blue}{crArgPos} (int {\bf valor});

\textcolor{red}{Funciones para la manipulaci�n de las LANS}
int \textcolor{blue}{creaLans} (int {\bf d});
int \textcolor{blue}{fusionaLans} (int {\bf x}, int {\bf y});
void \textcolor{blue}{completaLans} (int {\bf x}, TIPO_ARG {\bf arg});
\end{alltt}
\end{small}
\end{blistas}

\end{traspa}
%%%%%%%%%%%%%%%%%%%%%%%%%%%%%%%%%%%%%%%%%%%%%%%%%%%%%%%%%%%%%%%%%%%%%%%%
\begin{traspa}{270}{ejemplo de generaci�n de c�digo}

\vspace{5mm}
\begin{alltt}
operadorAditivo
  : MAS_         \{ \$\$ = ESUM; \}
  | MENOS_       \{ \$\$ = EDIF; \} ;
expresionAditiva
  : expresionMultiplicativa        \{ \$\$ = \$1; \}
  | expresionAditiva operadorAditivo expresionMultiplicativa
    \{
      \$\$.tipo = T\_ERROR;
      if (\$1.tipo == \$3.tipo == T_ENTERO) \$\$.tipo = T_ENTERO;
      else yyerror("Error de tipos en la `expresi�n aditiva'");

      \$\$.pos = creaVarTemp();
      /*************** Expresi�n a partir de un operador aritm�tico */
      emite(\$2, crArgPos(\$1.pos), crArgPos(\$3.pos), crArgPos(\$\$.pos));
    \} ;
\end{alltt}

\end{traspa}
%%%%%%%%%%%%%%%%%%%%%%%%%%%%%%%%%%%%%%%%%%%%%%%%%%%%%%%%%%%%%%%%%%%%%%%%
\begin{traspa}{270}{ejemplo de uso}

\vspace{-10mm}
\begin{blistas}{10}{5}
\item \textcolor{blue}{\Large Ejemplo de programa de c�digo intermedio}

\vspace{-10mm}
\begin{alltt}
// Calcula el factorial de un m�mero > 0 y < 13
//---------------------------------------------
int factorial (int n)
\{ int f;
  if (n <= 1) f=1;
  else f= n * factorial(n-1);
  return f;
\}
int main ()
\{ int x;
  read(x);
  if (x > 0) 
    if (x < 13) print(factorial(x)); 
    else \{\}
  else \{\}
  return 0;
\}
\end{alltt}
\end{blistas}

\end{traspa}
%%%%%%%%%%%%%%%%%%%%%%%%%%%%%%%%%%%%%%%%%%%%%%%%%%%%%%%%%%%%%%%%%%%%%%%%
\begin{traspa}{270}{ejemplo de GCI}

\vspace{-5mm}
\begin{scriptsize}
\begin{center}
\begin{tabular}
{|rll@{ , }l@{ , }l|p{5mm}|rll@{ , }l@{ , }l| }\cline{1-5}\cline{7-11}
  0&INCTOP&             &            &  i:   0    && 30&PUSHFP&             &            &            \\
  1&GOTOS &             &            &  e:  30    && 31&FPTOP &             &            &            \\
  2&PUSHFP&             &            &            && 32&INCTOP&             &            &  i:  10    \\
  3&FPTOP &             &            &            && 33&EREAD &             &            &  p: (1,  0)\\
  4&INCTOP&             &            &  i:  12    && 34&EASIG &   p: (1,  0)&            &  p: (1,  1)\\
  5&EASIG &   p: (1, -3)&            &  p: (1,  1)&& 35&EASIG &   i:   0    &            &  p: (1,  2)\\
  6&EASIG &   i:   1    &            &  p: (1,  2)&& 36&EASIG &   i:   1    &            &  p: (1,  3)\\
  7&EASIG &   i:   1    &            &  p: (1,  3)&& 37&EMAY  &   p: (1,  1)&  p: (1,  2)&  e:  39    \\
  8&EMENEQ&   p: (1,  1)&  p: (1,  2)&  e:  10    && 38&EASIG &   i:   0    &            &  p: (1,  3)\\
  9&EASIG &   i:   0    &            &  p: (1,  3)&& 39&EIGUAL&   p: (1,  3)&  i:   0    &  e:  55    \\
 10&EIGUAL&   p: (1,  3)&  i:   0    &  e:  14    && 40&EASIG &   p: (1,  0)&            &  p: (1,  4)\\
 11&EASIG &   i:   1    &            &  p: (1,  4)&& 41&EASIG &   i:  13    &            &  p: (1,  5)\\
 12&EASIG &   p: (1,  4)&            &  p: (1,  0)&& 42&EASIG &   i:   1    &            &  p: (1,  6)\\
 13&GOTOS &             &            &  e:  25    && 43&EMEN  &   p: (1,  4)&  p: (1,  5)&  e:  45    \\
 14&EASIG &   p: (1, -3)&            &  p: (1,  5)&& 44&EASIG &   i:   0    &            &  p: (1,  6)\\
 15&EPUSH &             &            &  i:   0    && 45&EIGUAL&   p: (1,  6)&  i:   0    &  e:  54    \\
 16&EASIG &   p: (1, -3)&            &  p: (1,  6)&& 46&EPUSH &             &            &  i:   0    \\
 17&EASIG &   i:   1    &            &  p: (1,  7)&& 47&EASIG &   p: (1,  0)&            &  p: (1,  7)\\
 18&EDIF  &   p: (1,  6)&  p: (1,  7)&  p: (1,  8)&& 48&EPUSH &             &            &  p: (1,  7)\\
 19&EPUSH &             &            &  p: (1,  8)&& 49&CALL  &             &            &  e:   2    \\
 20&CALL  &             &            &  e:   2    && 50&DECTOP&             &            &  i:   1    \\
 21&DECTOP&             &            &  i:   1    && 51&EPOP  &             &            &  p: (1,  8)\\
 22&EPOP  &             &            &  p: (1,  9)&& 52&EWRITE&             &            &  p: (1,  8)\\
 23&EMULT &   p: (1,  5)&  p: (1,  9)&  p: (1, 10)&& 53&GOTOS &             &            &  e:  54    \\
 24&EASIG &   p: (1, 10)&            &  p: (1,  0)&& 54&GOTOS &             &            &  e:  55    \\
 25&EASIG &   p: (1,  0)&            &  p: (1, 11)&& 55&EASIG &   i:   0    &            &  p: (1,  9)\\
 26&EASIG &   p: (1, 11)&            &  p: (1, -4)&& 56&EASIG &   p: (1,  9)&            &  p: (1, -3)\\
 27&TOPFP &             &            &            && 57&TOPFP &             &            &            \\
 28&FPPOP &             &            &            && 58&FPPOP &             &            &            \\
 29&RET   &             &            &            && 59&FIN   &             &            &            \\
\cline{1-5}\cline{7-11}
\end{tabular}
\end{center}
\end{scriptsize}

\end{traspa}
%%%%%%%%%%%%%%%%%%%%%%%%%%%%%%%%%%%%%%%%%%%%%%%%%%%%%%%%%%%%%%%%%%%%%%%%
\end{document}
%%%%%%%%%%%%%%%%%%%%%%%%%%%%%%%%%%%%%%%%%%%%%%%%%%%%%%%%%%%%%%%%%%%%%%%%
